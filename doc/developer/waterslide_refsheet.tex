%% Single page reference to WATERSLIDE (WS) commands and 'kids'.

%% This produces the correct format, landscape .pdf output:
%% pdflatex waterslide_refsheet.tex
%% acroread waterslide_refsheet.pdf

%% See also the Makefile in the WATERSLIDE code development repository
%% doc/developer/

\documentclass[landscape]{article}
\usepackage[letterpaper,landscape]{geometry}
\usepackage{multicol}
\usepackage{array}
\usepackage{fancyhdr}
\usepackage{url}

%% Page dimensions, for letter paper in landscape orientation:
\geometry{top=0.75in,left=0.5in,right=0.5in,bottom=0.75in}

%% Put info in the header and footer.
\pagestyle{fancy}
\lhead{} \rhead{}
\chead{}
\lfoot{} \rfoot{}
\cfoot{}
\renewcommand\headrulewidth{0.5pt}
\renewcommand\footrulewidth{0.5pt}

%% Constants:
\def\stdversion{1.0.0}          %Version of WS
\def\stddate{17~March~2016}     %Date of the WS release
\def\qrversion{0.1}             %Version of this quick reference.

\textwidth=10in

%% Definition for lettersize paper, three columns:
\newdimen\intercolumnskip	%horizontal space between columns
\def\scaledmag#1{ scaled \magstep #1}
\font\titlefont=cmbx10 \scaledmag2
\font\headingfont=cmbx10 \scaledmag1
\font\subheadingfont=cmbx9
\font\smallfont=cmr6
\font\smallsy=cmsy6
\font\eightrm=cmr8
\font\eightbf=cmbx8
\font\eightit=cmti8
\font\eighttt=cmtt8
\font\eightmi=cmmi8
\font\eightsy=cmsy8
\textfont0=\eightrm
\textfont1=\eightmi
\textfont2=\eightsy
\def\rm{\eightrm}
\def\bf{\eightbf}
\def\it#1{\begin{itshape}#1\end{itshape}}
\def\tt#1{\texttt{#1}}
\normalbaselineskip=.8\normalbaselineskip
\normallineskip=.8\normallineskip
\normallineskiplimit=.8\normallineskiplimit
\normalbaselines\rm		%make definitions take effect
\intercolumnskip=.46in
\parindent 0pt
\parskip 1ex plus .5ex minus .5ex
%% Gives tighter packing:
%% \setlength{\parskip}{0pt plus 0.5ex} 

%% Custom definitions:
\newcommand{\command}[1]{\bf{\tt{#1}}}
\newcommand{\cmd}[2]{\command{\url+#1+}\rm{~-~#2}\hangindent=10pt\par}

\def\small{\smallfont\textfont2=\smallsy\baselineskip=.8\baselineskip}

% newcolumn - force a new column.  Use sparingly, probably only for
% the first column of a page, which should have a title anyway.
\outer\def\newcolumn{\vfill\eject}

% title - page title.  Argument is title text.
\outer\def\title#1{{\titlefont\centerline{#1}}\vskip 1ex plus .5ex}

%% Redefine section comamnds to use less space
\makeatletter
\renewcommand{\section}{\@startsection{section}{1}{0mm}%
  {-1ex plus -.5ex minus -.2ex}%
  {0.5ex plus .2ex}%x
  {\normalfont\large\bfseries}}
\renewcommand{\subsection}{\@startsection{subsection}{2}{0mm}%
  {-1ex plus -.5ex minus -.2ex}%
  {0.5ex plus .2ex}%
  {\normalfont\normalsize\bfseries}}
\renewcommand{\subsubsection}{\@startsection{subsubsection}{3}{0mm}%
  {-1ex plus -.5ex minus -.2ex}%
  {1ex plus .2ex}%
  {\normalfont\small\bfseries}}
\makeatother

%% Don't print section and subsection numbers:
\setcounter{secnumdepth}{0}

% paralign - begin paragraph containing an alignment.
% If an \halign is entered while in vertical mode, a parskip is never
% inserted.  Using \paralign instead of \halign solves this problem.
\def\paralign{\vskip\parskip\halign}

% \<...> - surrounds a variable name in a code example
\def\<#1>{{\it #1\/}}

% kbd - argument is characters typed literally.  Like the Texinfo command.
\def\kbd#1{{\tt#1}\null}	%\null so not an abbrev even if period follows

% beginexample...endexample - surrounds literal text, such a code example.
% typeset in a typewriter font with line breaks preserved
\def\beginexample{\par\leavevmode\begingroup
  \obeylines\obeyspaces\parskip0pt\tt}
{\obeyspaces\global\let =\ }
\def\endexample{\endgroup}

\begin{document}
\raggedright
%% Set the overall font size here:
\footnotesize
\begin{multicols}{3}

\title{WATERSLIDE Kids}
\centerline{(for version \stdversion)}

This is a single-page quick reference to the WATERSLIDE (WS)
stream processing system, and its modules, called `kids.'  It conforms
to version \stdversion, published on \stddate. This document is
version \qrversion. It is stored in the WS code development
repository, at \path{doc/developer/waterslide_refsheet.tex}.

To see further documentation, try
`\tt{wsman KID}' or `\tt{wsman -v KID}' or `\tt{wsman 
  --help}'.

In general, the ordering of the kids within the groups below is
from most commonly used to least commonly used.


\section{Input and Output}

\cmd{csv_in}{a source for event tuples based on character-delimited
  data}

\cmd{file_in}{reads files as memmap-ed binary
  buffers}

\cmd{wsproto_in}{reads data or files from stdin in Protocol Buffer
  formats (wsproto and pbmeta), creates metadata}

\cmd{print}{prints tuple data to STDOUT or a file}

\cmd{wsproto_out}{prints metadata in Protocol Buffer format}


\section{Data and Tuple Manipulation}

\cmd{subtuple}{selects data based on presence of tuple members}

\cmd{removefromtuple}{specifies items by label in a tuple to be
  removed from a tuple. The emitted tuple will not contain the
  specified items.}

\cmd{tuplehash}{generates a hash based on specific items in a
  tuple. This is used to combine data from different data fields into
  a single key for state tracking algorithms.}

\cmd{splittuple}{splits tuple into multiple tuples based on labels. If
  multiple items exist in a tuple, it will create a tuple for each
  item. It is also possible to specify data to keep/replicate for each
  new tuple.}

\cmd{mergetuple}{merges data from multiple tuples based on a shared,
  specified key}

\cmd{mklabelset}{create a label set from labels of tuple members}

\cmd{appendfirstitem}{appends item to tuple based on key, useful for
  inference and merging}

\cmd{appendlast}{appends last item to tuple based on key, useful for
  inference and merging}


\section{Counting}

\cmd{bandwidth}{tracks items per second, counts all items. Reports
  output at end of stream}

\cmd{keycount}{counts number of items with a specific key value. For
  instance the number of events with name ABCD}

\cmd{keyadd}{adds values of a given labeled data field and accumulates
  these sums based on key}

\cmd{keyaverage}{computes the average value of a data field based on
  sums of the same key}

\cmd{labelstat}{counts the occurrence of each label within tuples}


\section{Numeric Calculation}

\cmd{calc}{performs numerical calculations. Can also act as a filter
  based on numerical results.}


\section{Filtering}

\cmd{uniq}{determines if labeled items are new. Multiple ports allows
  for removal and queries against the state of each labeled key.}

\cmd{uniqexpire}{determines if items are new, expires out old items
  based on timestamp}

\cmd{firstn}{selects the first N tuples from each key}

\cmd{mostnew}{declares the first N items as new}

\cmd{bloom}{determines if items are new, keep track of items using a
  bloom filter}

\cmd{sample}{samples items based on probability}

\cmd{cntquery}{determine if the value of a keys is mostly
  positive. This module has multiple ports, one for INCREMENTing, one
  for DECREMENTing, one for querying. This is for finding items
  sequence conditions that are biased.}


\section{Matching}

\cmd{match}{finds strings in character buffers that match a dictionary
  of strings at arbitrary offset locations}

\cmd{fixedmatch}{finds strings in character buffer that match a
  dictionary of strings at fixed, explicit locations in the character
  buffer. Can also be used for protocol detection.}

\cmd{match_unit}{matches integers with specified properties and/or
  numeric ranges}

\cmd{equal}{checks to see if two elements in a tuple are equal}

\cmd{haslabel}{checks to see if a label or set of labels exists in an
  event or members of a tuple}

\cmd{re2}{when WS is compiled with Google’s re2 library, this allows
  you to specify perl compatible regular expressions including
  extracting content}


\section{State Tracking}

\cmd{uniq}{only passes one uniq item. Items selected by label.}

\cmd{appendfirstitem}{stores first item (value) at a key, subsequent
  queries of the key results in the stored value appended to the query
  tuple}

\cmd{appendlast}{stores last item (value) at a key, subsequent queries
  of the key results in the stored value appended to the query tuple}

\cmd{keyflow}{creates a temporal hash based on keys that occur close
  in time}

\cmd{keeplast}{stores the last data item at the key. Only outputs data
  when table is full, or the table is flushed.}

\cmd{cntquery}{determines of state of key has reached a counting
  threshold. Uses ports to increment or decrement state. Can also be
  used to check for imbalance of events, such as HTTP server hostnames
  occurring without a reference.}

\cmd{firstn}{passes the first n events for each key}

\cmd{keepn}{keeps the last n items at the key}

\cmd{stateclimb}{user specifies labels and values (positive or
  negative). when label is found the value is added to accumulator for
  each key. If accumulator reaches a target value, then output is
  triggered.}


\section{Decoding}

\cmd{base64}{takes an identified base64 encoded buffer and decodes it
  into another buffer}

\cmd{asciihex}{takes an ASCII string with hex values and converts them
  to binary}

\section{Distributed Processing}

\cmd{tcpthrow}{writes TCP messages, sends to connected client}

\cmd{tcpcatch}{reads from TCP server, creates binary data for parsing}

\section{Miscellaneous Commands}

\cmd{flush}{resets state hash tables, based on number of events seen
  or elapsed time}

\rule{0.3\linewidth}{0.25pt}
\scriptsize

\end{multicols}
\end{document}
